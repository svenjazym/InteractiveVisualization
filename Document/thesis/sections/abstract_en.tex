%% LaTeX2e class for student theses
%% sections/abstract_en.tex
%% 
%% Karlsruhe Institute of Technology
%% Institute for Program Structures and Data Organization
%% Chair for Software Design and Quality (SDQ)
%%
%% Dr.-Ing. Erik Burger
%% burger@kit.edu
%%
%% Version 1.3.3, 2018-04-17

\Abstract
A fundamental task of Data Mining is to estimate the correlation between the attributes of a data set. Knowing the relationship between a set of variables, one can infer useful knowledge about external, a priori unknown outcomes.\\
In contrast to static data, the data is often available as a stream, i.e., it is an infinite, ever evolving sequence of observations. Concepts learned at a certain time cannot be expected to hold in the future. Therefore, correlation estimation should be a continuous process.\\
Also, the data is often high-dimensional, i.e., it contains hundreds or thousands of dimensions. Besides the computational burden to estimate the correlation between many subsets, it becomes difficult for a human observer to extract knowledge from the results. The task becomes even more difficult if one considers correlations between more than two variables, because the size of the result increases exponentially.
\\
The aim of this bachelor thesis is to develop a graphical interface for data scientists, dedicated to the visualization of correlation in user-given data streams. With this interface, available for example as a web-service, users may provide their own data sets. Then, the system’s back-end estimates the correlations and  visualizations of the results. Users interact in several ways with the interface, by setting parameters to tune the visualization. We evaluate the benefits of our interface and determine which visualization is the most appropriate, depending on specific types of user query, via controlled user studies.

